%\documentstyle[epsf,twocolumn]{jarticle}       %LaTeX2.09�d�l
\documentclass[twocolumn]{jarticle} 
\setlength{\topmargin}{-45pt}
%\setlength{\oddsidemargin}{0cm} 
\setlength{\oddsidemargin}{-7.5mm}
%\setlength{\evensidemargin}{0cm} 
\setlength{\textheight}{24.1cm}
%setlength{\textheight}{25cm} 
\setlength{\textwidth}{17.4cm}
%\setlength{\textwidth}{172mm} 
\setlength{\columnsep}{11mm}

% 【節が変わるごとに (1.1)(1.2) … (2.1)(2.2) と数式番号をつけるとき】
%\makeatletter
%\renewcommand{\theequation}{%
%\thesection.\arabic{equation}} %\@addtoreset{equation}{section}
%\makeatother

%\renewcommand{\arraystretch}{0.95}行間の設定

%%%%%%%%%%%%%%%%%%%%%%%%%%%%%%%%%%%%%%%
\usepackage[dvipdfmx]{graphicx} %pLaTeX2e仕様(\documentstyle ->\documentclass)
\usepackage{url}		%参考文献にurlを入れる用
\usepackage{bm}  	%太字形式のベクトルを使う用
\usepackage{amsmath}	%数式の場合分け用
%%%%%%%%%%%%%%%%%%%%%%%%%%%%%%%%%%%%%%%

\begin{document}

	%bibtex用の設定
	%\bibliographystyle{ujarticle}

	\twocolumn[
		\noindent
		\hspace{1em}
		2021 年 5 月 28 日
		研究会資料
		\hfill
		B4	尾關  拓巳
		\vspace{2mm}
		\hrule
		\begin{center}
			{\Large \bf 進捗報告}
		\end{center}
		\hrule
		\vspace{9mm}
	]

\section{今週やったこと}
	\begin{itemize}
		\item 線形計画問題の理解
		\item 線形計画ソルバの調査
		\item Pyomoの調査
	\end{itemize}

\section{線形計画問題の理解}
    線形計画問題とは,決定変数が連続変数で,制約条件や目的関数が全て線形の式で表現された数理計画問題である.線形の式は変数の一次式で記述してある式のことである.

\section{線形計画ソルバの調査}
    先週の研究会でいただいた資料「使ってみよう線形計画ソルバ」\cite{40021875158}を読んだ.線形計画法のソルバやその使い方を主として紹介しているが,非線形や混合整数の計画問題も解けるソルバの紹介も含まれていた.
    
    ここで,電気学会の最適化ベンチマーク問題集\cite{denki}より,ベンチマーク問題は混合整数非線形計画問題(MINLP)であり,以下の特徴がある.
    \begin{itemize}
        \item 制約条件の多くが決定変数に対する上下限制約である.
        \item 離散変数は各機器の連続運転,停止時間を表現するために導入されている.
    \end{itemize}
    よって,これらをうまく対処するソルバを探したいと考えた.

    \cite{40021875158}で紹介されている線形計画ソルバには,商用ソルバと,非商用/アカデミックソルバがあり,それぞれ調査をした.結果としては非商用ソルバでは混合整数計画問題は解けるが,非線形計画問題に対応したソルバは見つからなかった.商用ソルバは非線形計画問題に対応しているものも多く見られた.
    
    Gurobi Optimizer\cite{gurobi}のサイトには非線形計画問題に対応しているとは書かれていなかったものの三菱との会議の際にベンチマーク問題への成果を伺っているため疑問が残った.

    FICO Xpress Optimization\cite{fico}のサイトにはベンチマーク問題のが属するMINLPに対応していると具体的に記述があった.

    商用ソルバの中にはアカデミック版が存在するものもあるが,制限を受けるものが多かった.
    
    \cite{40021875158}で紹介されていないMINLPを解けるソルバとしてAIMMS Outer Approximation(AOA)\cite{AOA}があった.このソルバはアカデミック版での制限はないと記載されていた.

\section{Pyomoの調査}
    Pyomo\cite{bynum2021pyomo}\cite{hart2011pyomo}はpythonベースのオープンソースの最適化モデリング言語で,多様な最適化機能を備えている.
    \subsection{MindtPy}
        MindtPy(Mixed-Integer Nonlinear Dicomposition Toolbox for Pyomo)\cite{BERNAL2018895}はPyomoで開発されたソフトウェアフレームワークで,分解アルゴリズムを用いてMINLPを解くことができる.MINLPの分解アルゴリズムは,基本的に混合整数線形計画法(MILP)や非線形計画法(NLP)に依存している.MindtPyはOuter Approximation(OA)アルゴリズム\cite{10.5555/2990007.2990122}とExtended Cutting Plane(ECP)アルゴリズム\cite{WESTERLUND1995131}が実装されている.

\section{今後の展望}
    \begin{itemize}
        \item pyomoのMindPyを用いてベンチマーク問題を解く
    \end{itemize}

% 参考文献
\bibliography{hoge}				%hogeはbibファイルのファイル名
\bibliographystyle{junsrt}		%順番に表示

\end{document}