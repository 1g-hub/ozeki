%\documentstyle[epsf,twocolumn]{jarticle}       %LaTeX2.09�d�l
\documentclass[twocolumn]{jarticle} 
\setlength{\topmargin}{-45pt}
%\setlength{\oddsidemargin}{0cm} 
\setlength{\oddsidemargin}{-7.5mm}
%\setlength{\evensidemargin}{0cm} 
\setlength{\textheight}{24.1cm}
%setlength{\textheight}{25cm} 
\setlength{\textwidth}{17.4cm}
%\setlength{\textwidth}{172mm} 
\setlength{\columnsep}{11mm}

% 【節が変わるごとに (1.1)(1.2) … (2.1)(2.2) と数式番号をつけるとき】
%\makeatletter
%\renewcommand{\theequation}{%
%\thesection.\arabic{equation}} %\@addtoreset{equation}{section}
%\makeatother

%\renewcommand{\arraystretch}{0.95}行間の設定

%%%%%%%%%%%%%%%%%%%%%%%%%%%%%%%%%%%%%%%
\usepackage{graphicx} %pLaTeX2e仕様(\documentstyle ->\documentclass)
\usepackage{url}		%参考文献にurlを入れる用
\usepackage{bm}  	%太字形式のベクトルを使う用
%%%%%%%%%%%%%%%%%%%%%%%%%%%%%%%%%%%%%%%

\begin{document}

	%bibtex用の設定
	%\bibliographystyle{ujarticle}

	\twocolumn[
		\noindent
		\hspace{1em}
		2021 年 5 月 7 日
		研究会資料
		\hfill
		B4	尾關  拓巳
		\vspace{2mm}
		\hrule
		\begin{center}
			{\Large \bf 進捗報告}
		\end{center}
		\hrule
		\vspace{9mm}
	]

\section{今週やったこと}
\begin{itemize}
  \item CMA-ESでベンチマークを解く
\end{itemize}
	
\section{ネルダーミード法でベンチマークを解く}
	\subsection{実験設定}
	CMA-ESの実験設定を表1に示す.
	\begin{table}[htbp]
		\begin{center}
			\caption{CMA-ESの実験設定}
			\begin{tabular}{| c | c |} \hline
				最大世代数 & 3000 \\
 				$\sigma$ & 3.0 \\
				$\lambda$ & 2400 \\ \hline
				
			\end{tabular}
		\end{center}
	\end{table}
	
	\subsubsection{評価関数と制約違反}
	取り扱うベンチマーク問題では制約違反を考慮する必要があり,今実験ではその許容量を$1.0\times10^{-10}$とする.また,制約違反の合計値$V$を式1のようにCMA-ESの目的関数$F$に組み込む.
	\begin{equation}
		F(\bm{x}) = f(\bm{x}) +  10^{16}V 
	\end{equation}
	ただし,ベンチマークの目的関数を$f$とする.
	
	\subsection{結果}
	上記の実験の結果を表2に示す.
	\begin{table}[htbp]
		\begin{center}
			\caption{実験結果}
			\begin{tabular}{| c | c |} \hline
				目的関数値 & 制約違反 \\ \hline 
				4225738.328 &  4667.910409 \\ \hline
			\end{tabular}
		\end{center}
	\end{table}

	制約違反が大きく許容量を超えており,目的関数の最小化も上手くいっていない.パラメーターを変えるなど予備実験を行ったがその結果は同等であった.120次元という次元数の高い問題にネルダーミード法が適していないことが原因ではないかと考えた.

	
\section{今後の予定}
% なんとなくなんかの勉強をするとかではなく具体的に

\begin{itemize}
  \item 他の最適化手法を試す.
\end{itemize}

% 参考文献
\bibliography{hoge}				%hogeはbibファイルのファイル名
\bibliographystyle{junsrt}		%順番に表示

\end{document}
